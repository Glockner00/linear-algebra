\documentclass{report}

\input{preamble}
\input{macros}
\input{letterfonts}
\usepackage{blkarray}


\title{\Huge{TATA24}\\ Gauselimination}
\author{\huge{Axel Glöckner}}
\date{\today}
\begin{document}
\maketitle


\dfn{Linjära ekvationssystem}{
	\begin{align*}
		a_1x_1 + a_2x_2 + \ldots + a_nx_n & =b &  & a_1\ldots a_n, b \in R \\
	\end{align*}
}

\nt{
	\begin{align*}
		x + 2y -3z & = 17, & \textbf{linjär}      \\
		x^2 + 3    & = y,  & \textbf{icke linjär} \\
	\end{align*}
}

\ex{Gauselimination}{
	\begin{cases}
		x + y + 2z = 2, & \ {-4} / {-3} \to \text{ekv}_2 / \text{ekv}_3 \\
		4x + 6y + 10z = 14                                              \\
		3x + 4y + 9z = 13
	\end{cases}
	\Leftrightarrow &
	\begin{cases}
		x + y + 2z = 2                            \\
		0 + 2y + 2z = 6, & {1/2} \to \text{ekv}_3 \\
		0 + y + 3z = 7
	\end{cases}
	\\ \\ \\
	\Leftrightarrow &
	\begin{cases}
		x + y + 2z = 2 \\
		0 + y + z = 3  \\
		0 + 0 + 2z = 4
	\end{cases}
	\Leftrightarrow &
	\begin{cases}
		x = -3 \\
		y = 1  \\
		z = 2
	\end{cases}
}

\ex{Gauselimination syntax}{
\begin{pmatrix}
    1 & 1 & 2 & 2 \\
    4 & 6 & 10 & 14 \\
    3 & 4 & 9 & 13
\end{pmatrix}
\begin{array}{l}
    \text{\circled{-4} \rotatebox[origin=c]{180}{&\Rsh&} and \circled{-3}\rotatebox[origin=c]{180}{&\Rsh&}\rotatebox[origin=c]{180}{&\Rsh&}} \\
    \text{} \\
    \text{}
\end{array}
\sim \quad
\begin{pmatrix}
    1 & 1 & 2 & 2 \\
    0 & 2 & 2 & 6 \\
    0 & 1 & 3 & 7
\end{pmatrix}
\begin{array}{l}
    \\
    \text{\circled{1/2} \rotatebox[origin=c]{180}{&\Rsh&}  } \\
    \\ 
\end{array}
& \sim & \quad
\begin{pmatrix}
    1 & 1 & 2 & 2 \\
    0 & 2 & 2 & 6 \\
    0 & 0 & 2 & 4 \\
\end{pmatrix}
\begin{array}{l}
    \\
    \\
    \\
\end{array} 
& \Leftrightarrow \quad
\begin{cases}
   x = -3 \\
   y = 1 \\
   z = 2\\
\end{cases}
}

\dfn{Elementära radoperationer}{
    \begin{itemize}
    \item Addera en multipel av en rad till en annan. 
    \item Multiplicera en rad med en konstan som är \neq 0 
    \item Byta plats på två rader 
    \end{itemize}
  
}


\dfn{}{
    Trappstegsmatris,
    \begin{pmatrix}
        \circled{1} & 1 & 2 & 2 \\
        0 & \circled{1} & 1 & 3 \\
        0 & 0 & \circled{2} & 4 \\
    \end{pmatrix}
    \quad leder till
    \begin{cases}
        \circled{x} + y + 2z &= 2 \\
        0 + \circled{y} + z &= 3 \\
        0 + 0 +  \circled{2z} &= 4 
    \end{cases}
    \quad \Leftrightarrow
    \begin{cases}
        x &= -3 \\
        y &= 1 \\
        z &= 2
    \end{cases}
    
    \begin{itemize}
        \item De inringade elementen kallas för \textbf{Pivotelement}.
        \item Trappstegsmatris
            \subitem Eventuella noll-rader står längst ned.
            \subitem Det förta nollskillda elementet i en rad står längre till höger i rader som står länger ned.

        \item Gauselimination
            \subitem Utför elementära radoperationer tills man har en trappsstegasmatris. Detta går \textbf{alltid}. 
    \end{itemize}
}

\ex{Gauselimination}{  
    \begin{align*}
            \begin{cases}
                x_2 + x_3 &= -2 \\
                x_1 + 2x_2 + 3x_3 &= -4\\
                x_1 - 3x_3 &= 0
            \end{cases}
    .\end{align*}
}

\sol{
    \\ \\ \\
    \begin{pmatrix}
        0 & 1 & 3 & -2 \\
        1 & 2 & 3 & -4 \\
        1 & 0 & -3 & 0 \\
    \end{pmatrix}
    \begin{array}{l}
        \\
        \\
        \text{\circled{-1} & \rotatebox[origin=c]{}{&\Lsh&} }  \\
    \end{array}
    \quad \sim & 
    \begin{pmatrix}
        0 & 1 & 3 & -2 \\
        0 & 2 & 6 & -4\\
        1 & 0 & -3 & 0 \\
    \end{pmatrix}
    \begin{array}{l}
        \text{\circled{-2} \rotatebox[origin=c]{180}{&\Rsh&}}\\
        \\
        \\
    \end{array}
    \quad \sim &
    \begin{pmatrix}
        0 & 1 & 3 & -2 \\
        0 & 0 & 0 & 0 \\
        1 & 0 & -3 & 0 \\
    \end{pmatrix}
    \begin{array}{l}
        \updownarrow\\
        \updownarrow\\
        \\
    \end{array}
    \quad \sim & 
    \begin{pmatrix}     
        1 & 0 & -3 & 0 \\
        0 & 1 & 3 & -2 \\
        0 & 0 & 0 & 0
    \end{pmatrix} 
    \\ \\ \\  \quad Som motsvarar
    \begin{cases}
        x_1 - 3x_3 &= 0 \\
        x_2 + 3x_3 &= 2 \\
        0 &= 0
    \end{cases}
    \quad \Leftrightarrow
    \begin{cases}
        x_1 &= 3t \\
        x_2 &= -2 - 3t \\
        x_3 &= t 
    \end{cases}
    \begin{array}{l}
        \\
        \\
        t \in R
    \end{array}{}
    detta kallas för en \textbf{parameterlösning.}
}

\nt{
    För lösningsmängden gäller endera av följande, \\ 
    \begin{array}
        \text{Unik-lösning} \\ \\
        \begin{pmatrix}
            \circled{1} & 2 & -3 & 0\\
            0 & \circled{2} & 3 & -1 \\
            0 & 0 & \circled{-2} & 1 \\
        \end{pmatrix} \\
        \text{Ett PE i varje kolumn}\\
        \text{i VL, inget i HL}
    \end{array}
    \quad
    \begin{array}
        \text{Oändligt många lösningar}\\ \\
        \begin{pmatrix}
            \circled{1} & 0 & -3 & 0 \\
            0 & \circled{1} & 3 & -2 \\
            0 & 0 & 0 & 0
        \end{pmatrix}\\
        \text{Kolumner saknar}\\
        \text{PE i VL inget iHL}
    \end{array}
    \quad 
    \begin{array}
        \text{ingen lösning} \\ \\
        \begin{pmatrix} 
        \circled{1} & 0 & -3 & 0 \\
        0 & \circled{1} & 3 & -2 \\
        0 & 0 & 0 & \circled{-1}
        \end{pmatrix} 
        \\
        \text{PE i HL}        
    \end{array}
    }
        
    \ex{}{
    \text{Ange antalet lösningar till,} &
    \begin{cases}
        x - ay = 0, &\text{ för alla a \in & R}\\
        ax - 3ay = 0
    \end{cases}
}

\sol{
    \\ \\
    \begin{pmatrix}
        1 & -a & -b \\
        a & -3a & 0 \\
    \end{pmatrix}
    \begin{array}{l}
        \text{\circled{-a} \rotatebox[origin=c]{180}{&\Rsh&}} \\
        \\
    \end{array}
    \quad \sim &
    \begin{pmatrix}
        \circled{1} & -a & 0  \\
        0 & a^2-3a & -6a\\
    \end{pmatrix}
    \quad alltså, & 
    a^2-3a = 0 \Leftrightarrow 
    \quad 
    \begin{cases}
        a = 0\\
        a = 3
    \end{cases}
    \begin{array}{l}
        \text{oändligt många lösnignar}\\
        \text{olösligt}
    \end{array}{}
}

\rotatebox[origin=c]{180}{&\Rsh&}



\end{document}

